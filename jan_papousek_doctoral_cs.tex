\documentclass[11pt,a4paper]{moderncv}
\usepackage{cv}

\tlmaxdates{2008}{2015}
\linespread{0.9}

\firstname{Jan}
\familyname{Papoušek}
\address{Libušino údolí 152}{623 00 Brno}
\email{jan.papousek@gmail.com}
\photo[70pt]{jan_papousek.jpeg}
\extrainfo{\linkedin~\httplink{www.linkedin.com/in/janpapousek}\\\github~\httplink{www.github.com/papousek}\\\httplink{https://is.muni.cz/auth/osoba/325494}}

\begin{document}
\maketitle

\section{Vzdělání}
\tlcventry[timelinecolor]{2011}{0}{Navazující magisterské studium}{Fakulta informatiky Masarykovy univerzity}{Brno}{}{Obor studia: Paralelní a distribuované systémy}
\subsection{Analýza robustnosti spojitých dynamických systémů v distribuovaném prostředí}
\cvline{Diplomová práce}{Práce prezentuje algoritmus pro analýzu dynamických systémů zadaných pomocí soustavy obyčejných diferenciálních rovnic vzhledem k chování charakterizovanému vlastností temporální logiky signálů. Výsledkem analýzy je představa o tom, jakým způsobem ovlivňují změny modelu jeho chování. Popisovaná metoda se zakládá na již existujícím algoritmu a výpočtu lokální robustnosti.\resource{www.is.muni.cz/auth/th/325494/fi\_m}}

\medskip
\tlcventry[timelinecolor]{2008}{2011}{Bakalářské studium}{Fakulta informatiky Masarykovy univerzity}{Brno}{}{Obor studia: Paralelní a distribuované systémy}
\subsection{Paralelizace metod pro analýzu dynamických systémů pomocí grafické karty}
\cvline{Bakalářská práce}{Práce prezentuje postup, jak paralelizovat numerickou simulaci nad velkým množstvím iniciálních bodů v rámci existující metody pro analýzu dynamických systémů. Vychází ze známých metod pro numerickou simulaci, konkrétně Eulerova explicitního a Runge-Kutta-Fehlbergova schématu. Tyto metody byly implementovány tak, aby je bylo možné spustit na grafické kartě za použití technologie CUDA. \resource{www.is.muni.cz/th/325494}}

\section{Výzkumná činnost}
\tlcventry[timelinecolor]{2010}{0}{Laboratoř systémové biologie}{Fakulta informatiky Masarykovy univerzity}{Brno}{}{Vývoj nástroje pro analýzu dynamických systémů.\contact{Botanická 68a, 602 00 Brno}{+420 549 495 990}{sybila@fi.muni.cz}{sybila.fi.muni.cz}}

\subsection{Parasim -- Nástroj pro paralelní simulaci a verifikaci}
\cvline{Projekt}{Nástroj pro analýzu spojitých dynamických systémů modelovaných pomocí soustavy diferenciálních rovnic. Aplikace byla realizována v rámci studentského výzkumného projektu MUNI33/052012.  \resource{www.github.com/sybila/parasim}}

\section{Zkušenosti s výukou}
\tlcventry[timelinecolor]{2009}{2011}{Asistent ve výuce}{Fakulta informatiky Masarykovy univerzity}{Brno}{}{PB162 Programování v jazyce Java, IB108 Návrh algoritmů II and IB015 Úvod do funkcionálního programování\contact{Botanická 68a, 602 00 Brno, Czech Republic}{+420 549 491 810}{info@fi.muni.cz}{www.fi.muni.cz}}
\tlcventry[timelinecolor]{2010}{2012}{Vedoucí kroužku}{Středisko volného času Lužánky}{Brno}{}{Programování Java, Základy webových stránek (HTML a CSS)\contact{Lidická 50, 658 12 Brno}{+420 776 204 769}{pspatka@luzanky.cz}{www.fi.muni.cz}}

\section{Aktivity směřující k propagaci informatiky}
\tlcventry[timelinecolor]{2008}{2012}{Korespondenční seminář}{Fakulta informatiky Masarykovy univerzity}{Brno}{}{Vymýšlení a opravování úloh z oblasti informatiky určených pro studenty středních škol.\resource{ksi.fi.muni.cz}}
\tlcventry[timelinecolor]{2009}{2012}{Intersob}{Masarykova univerzita}{Brno}{}{Spoluorganizování městské hry zaměřené na studenty středních škol.\resource{ks.math.muni.cz/intersob/}}
\tlcventry[timelinecolor]{2009}{2011}{Interlos}{Fakulta informatiky Masarykovy univerzity}{Brno}{}{Spoluorganizování internetové logické soutěže pro studenty středních a vysokých škol, která obsahuje mimo jiné různé úlohy z oblasti informatiky.\resource{interlos.fi.muni.cz}}

\section{Ostatní}
\cvline{\textbf{programování}}{Java\advanced, PHP\advanced, Bash\advanced, Scala\familiar, Haskell\familiar, R\familiar, Python\familiar, C~for~CUDA\familiar}
\cvline{\textbf{technologie}}{GIT\advanced, SVN\familiar, MySQL\advanced, \LaTeX\advanced, Beamer\advanced, Linux\advanced, Maven\advanced, Arquillian\contributor, Selenium\advanced}
\cvline{\textbf{teorie}}{formální verifikace, návrh algoritmů, paralelní výpočty}
\cvline{\textbf{jazyky}}{čeština, angličtina}
\cvline{\textbf{online kurzy}}{Computing for Data Analysis, Functional Programming Principles in Scala, Web Intelligence and Big Data, 	Algorithms -- Part I}

\devnotes{základní znalost}{aktivní používání}{přispěvatel}
\end{document}

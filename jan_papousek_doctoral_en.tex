\documentclass[11pt,a4paper]{moderncv}
\usepackage{cv}
\usepackage{bibentry}

\tlmaxdates{2008}{2018}
\linespread{0.9}

\firstname{Jan}
\familyname{Papoušek}
\address{Fleischnerova 20}{635 00 Brno, Czech Republic}
\email{jan.papousek@gmail.com}
\photo[70pt]{jan_papousek.jpeg}
\extrainfo{\linkedin~\httplink{www.linkedin.com/in/janpapousek}\\\github~\httplink{www.github.com/papousek}\\\httplink{is.muni.cz/osoba/jan.papousek?lang=en}}


\begin{document}
\maketitle

\vspace{-1cm}

\section{Education}

\medskip
\tlcventry[timelinecolor]{2013}{0}{Doctoral degree programme}{Faculty of Informatics Masaryk University}{Brno}{}{Field of study: Informatics}

\tlcventry[timelinecolor]{2011}{2013}{Master's degree programme}{Faculty of Informatics Masaryk University}{Brno}{}{Field of study: Parallel and Distributed Systems}
{\small
\cvline{}{\textbf{Robustness Analysis of Continuous-Time Dynamical Systems:} The work presents an algorithm for the analysis of dynamic systems specified by ordinary differential equations. The system behavior is characterized by the formula of signal temporal logic. The result of the analysis is the idea of how the changes affect the model's behavior.\resource{www.is.muni.cz/th/325494/fi\_m/?lang=en}}}

\medskip
\tlcventry[timelinecolor]{2008}{2011}{Bachelor's degree programme}{Faculty of Informatics Masaryk University}{Brno}{}{Field of study: Parallel and Distributed Systems}
{\small
\cvline{}{\textbf{Parallelization of Methods for Dynamic System Analysis Using CUDA:} The thesis presents a procedure of the parallelization a numerical simulation over a large amount of initial points. Methods for numerical simulations have been implemented so that they can be run on a graphics card while using the CUDA technology through the Java interface. \resource{is.muni.cz/th/325494/fi\_b/?lang=en}}}

\section{Research}
\tlcventry[timelinecolor]{2013}{0}{Adaptive Learning Group}{Faculty of Informatics Masaryk University}{Brno}{}{Systems for adaptive practice using machine learning techniques to analyze educational data, development of models predicting learners' performance\contact{Botanická 68a, 602 00 Brno, Czech Republic}{}{al@fi.muni.cz}{www.fi.muni.cz/adaptivelearning}}

%\cvline{}{\textbf{Outline Maps:} Computerized adaptive practice system for learning of geographical facts. The system is currently used more than a thousand of users per day and serves as a~platform for online multivariate testing. Based on the collected data we develop new predictive models and tune parameters of adaptive practice.\resource{www.slepemapy.cz/en/}}
%\cvline{}{\textbf{Practice Anatomy:} System similar to \textit{Outline Maps} providing practice of anatomical structures. Although the system has been launched recently, it is already used by a considerable number of medical students in Czech Republic and Slovakia.\resource{www.practiceanatomy.com}}

\tlcventry[timelinecolor]{2010}{2013}{Systems Biology Laboratory}{Faculty of Informatics Masaryk University}{Brno}{}{Development of a tool for analysis of dynamic systems\contact{Botanická 68a, 602 00 Brno, Czech Republic}{+420 549 495 990}{sybila@fi.muni.cz}{sybila.fi.muni.cz}}

\cvline{}{\textbf{Parasim, Tool for Paralell Simulations and Verification:} Parasim is a tool for robustness analysis. Given a SBML model, a property and perturbation set, it computes the robustness of model over the perturbation set with respect to the property.\resource{www.github.com/sybila/parasim}}

\section{Working Experience}
\tlcventry[timelinecolor]{2013}{2015}{Sofware Engineer}{GoodData}{Brno}{}{Development of a high level analytical language, which is compiled into an intermediate algebraic representation, optimized and then compiled into SQL.\contact{Lidická 965/31, 602 00 Brno, Czech Republic}{}{}{www.gooddata.com}}
\tlcventry[timelinecolor]{2011}{2013}{Quality Assurance Associate}{Red Hat Czech, Ltd.}{Brno}{}{Development of automated tests for RichFaces project, contributing to Arquillian project\contact{Purkyňova 99, 612 45 Brno, Czech Republic}{+420 532 294 111}{brno@redhat.com}{www.cz.redhat.com}}

\section{Teaching}
\tlcventry[timelinecolor]{2009}{2015}{Teaching Assistant}{Faculty of Informatics Masaryk University}{Brno}{}{Java, Algorithm Design II, Introduction to Functional Programming, Introduction to Programing using Python\contact{Botanická 68a, 602 00 Brno, Czech Republic}{+420 549 491 810}{info@fi.muni.cz}{www.fi.muni.cz}}

\section{Activities Aimed at Promoting of Computer Science}
\myproject{2008}{2012}{KSI}{co-organizer}{Correspondence Seminar from Informatics, under the auspices of Faculty of Informatics MU.}{ksi.fi.muni.cz}
\myproject{2009}{2012}{Intersob}{co-organizer and developer of evaluation system}{City game focused on high school students, under the auspices of MU.}{ganymed.math.muni.cz/intersob}
\myproject{2009}{2011}{Interlos}{co-organizer and developer of evaluation system}{Online programmer contest, under the auspices of Faculty of Informatics MU}{interlos.fi.muni.cz}

\section{Other}
\cvline{\textbf{programming}}{Python, Erlang, Java, PHP, Bash}
\cvline{\textbf{theory}}{machine learning, educational data mining, algorithm design, formal verification, parallel computing}
\cvline{\textbf{technologies}}{GIT, pandas, Linux, PostgreSQL, \LaTeX}
\cvline{\textbf{languages}}{Czech, English}
\cvline{\textbf{online courses}}{Computing for Data Analysis, Functional Programming Principles in Scala, Web Intelligence and Big Data, 	Algorithms -- Part I}

\nocite{umuai2016-elo}
\nocite{its2016-impact}
\nocite{lak2016-data-collection}
\nocite{lak2016-evaluation}
\nocite{aied2015-impact}
\nocite{aied2015-workshop-simulated}
\nocite{edm2015-response-times}
\nocite{edm2014-adaptive-facts}
\bibliographystyle{unsrt}
\bibliography{bibliography}

\end{document}
